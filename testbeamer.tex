
% diapositiva en blanc que genera pandoc perquè no hi ha dades de títol al md:
%\begin{frame}\frametitle{}
%
%\end{frame}

\section{Objectius}

\begin{frame}[fragile]\frametitle{Objectius}

Els \textbf{objectius} d'aquest presentació són:

\begin{itemize}[<+->]
\item
  Provar \emph{pandoc} per passar de \texttt{markdown} a
  \texttt{beamer}.
\item
  Estudiar el procés.
\item
  Introduir personalitzacions.
\item
  Publicar el resultat a
  \href{http://phobos.xtec.cat/jqueralt}{cataLàTeX}
\end{itemize}
\end{frame}

\section{Procediment}

\begin{frame}[fragile]\frametitle{Primer pas}

\begin{enumerate}[1.]
\item
  Crear un document de text en format \emph{markdown} anomenat
  \texttt{test.md}
\item
  Els títols de Nivell 1 (\texttt{\#}) seran les seccions de la
  presentació.
\item
  Els títols de Nivell 2 (\texttt{\#\#}) seran els títols de les
  diapositives.
\end{enumerate}
Sintaxi markdown: \url{http://daringfireball.net/projects/markdown/}

\end{frame}

\begin{frame}[fragile]\frametitle{Segon pas}

\begin{itemize}
\item
  Preparar un fitxer \texttt{matriu.tex} amb les especificacions per
  beamer.
\item
  Que cridi a fitxer \texttt{test.tex} on hi haurà el contingut de les
  diapositives.
\end{itemize}
Documentació de Beamer: \url{http://ctan.org/beamer}

\end{frame}

\begin{frame}[fragile]\frametitle{Tercer pas}

\begin{itemize}
\item
  Fer que \emph{pandoc} transformi \texttt{test.md} en \texttt{test.tex}
\end{itemize}
Comandament de pandoc:

\texttt{pandoc test.md -{}-slide-level 2 -t beamer -o test.tex}

\end{frame}

\begin{frame}[fragile]\frametitle{Quart pas}

\begin{itemize}
\item
  Crear el document PDF de la presentació.
\end{itemize}
Comandament de LaTeX:

\texttt{pdflatex matriu.tex}

\end{frame}

\begin{frame}[fragile]\frametitle{Pas alternatiu}

\begin{itemize}
\item
  Fer el pas directament des de pandoc.
\end{itemize}
Comandament de pandoc:

\texttt{pandoc -{}-toc -{}-slide-level 2 -V theme:NomTema -t beamer test.md -o test.pdf}

\end{frame}

\section{Resultat}

\begin{frame}\frametitle{}

\centering\includegraphics[scale=0.4]{fitxer.jpg}

\end{frame}

\begin{frame}\frametitle{Resultat}

El que esteu veient.

\end{frame}

\begin{frame}[fragile]\frametitle{Crèdits}

Idea original: \url{https://github.com/jeromyanglim}

Imatge amb llicència CC de
\href{http://flickrcc.bluemountains.net/flickrCC/index.php}{FlicrCC}

\end{frame}
